\section{Introduction}

\subsection{Purpose and Scope}

The HUPUF2X8A (Hutech PulseForge Stereo 2x8b Analog) is an audio synthesis chip designed for embedded systems requiring flexible and efficient sound generation.
This technical documentation provides comprehensive information on the architecture, configuration, and usage of the HUPUF2X8A APU.
It is intended for hardware developers, embedded software engineers, and audio system integrators who aim to incorporate the APU into audio processing applications.

\subsection{Features}

The HUPUF2X8A offers the following key features:
\begin{itemize}
    \item Dual-channel stereo output with 8-bit resolution per channel.
    \item Support for up to 8 audio channels:
    \begin{itemize}
        \item 6 Segmented Channels (2 High-Quality, 2 Medium-Quality, 2 Low-Quality).
        \item 2 Sampled Channels for raw audio playback.
    \end{itemize}
    \item Integrated waveform generation, including square, triangle, pulse, sawtooth, and noise.
    \item Modulation capabilities: Amplitude, Delta, Noise, and Period modulation.
    \item Dynamic mixing with individual channel volume control and stereo balancing.
    \item Flexible configuration via parallel register interface.
    \item Efficient handling of segmented waveforms, enabling the creation of complex audio patterns.
    \item Compact DIP-20 package with low power consumption.
\end{itemize}

\subsection{Scope of the Document}

This document covers the following aspects:
\begin{itemize}
    \item APU Architecture: Detailed block diagrams and functional module descriptions.
    \item Register Map: Comprehensive guide to configuring the APU.
    \item Performance Characteristics: Channel capacity, latency, and power consumption.
    \item Integration Guidelines: Recommendations for audio output circuitry and signal conditioning.
    \item Usage and Programming: Initialization routines and real-time control methods.
    \item Debugging and Testing: Diagnostic features and troubleshooting tips.
\end{itemize}

\subsection{Target Audience}

This document is designed for:
\begin{itemize}
    \item Audio hardware designers integrating the APU into embedded systems.
    \item Firmware developers writing control code for audio synthesis.
    \item System integrators looking to optimize audio performance.
    \item Quality assurance teams validating audio output quality and stability.
\end{itemize}

\subsection{Packaging and Pinout}

The HUPUF2X8A APU is available in a DIP-20 package. The table below describes the pin functions and directions:

\begin{table}[h!]
    \centering
    \begin{tabularx}{\linewidth}{|l|l|c|>{\raggedright\arraybackslash}X|l|}
        \hline
        \textbf{Pin(s)} & \textbf{Name} & \textbf{Direction} & \textbf{Description} & \textbf{Usage} \\
        \hline
        RA0-RA3 & Register Address & In & Register Access Address & Exposed Registers \\
        \hline
        RD0-RD7 & Register Data & In/Out & Register Access Data & Exposed Registers \\
        \hline
        RRW & Register R/W & In & Register Access Read/Write Control & Exposed Registers \\
        \hline
        RAE & Register Enable & In & Register Access Enable Signal & Exposed Registers \\
        \hline
        L & Left & Out & Sound Left Analog Signal & Sound Output \\
        \hline
        R & Right & Out & Sound Right Analog Signal & Sound Output \\
        \hline
        CLK & Master Clock & In & Master clock input & Clock \\
        \hline
        RST & Reset & In & Reset signal & Boot \\
        \hline
        VSS & Ground & - & Ground connection & Power supply \\
        \hline
        VDD & Power & - & Positive power supply & Power supply \\
        \hline
    \end{tabularx}
    \caption{Pinout of HUPUF2X8A - DIP-20 Package}
\end{table}

\begin{figure}[h!]
    \centering
    \begin{circuitikz} 
        % IC with 20 pins
        \draw (0,0) node[dipchip, num pins=20, external pad fraction=4](D) {\rotatebox{90}{HUPUF2X8A}};

        % Left Pins (1-10)
        \foreach \i/\pin in {1/VDD, 2/RA0, 3/RA1, 4/RA2, 5/RA3, 6/CLK, 7/RST, 8/RRW, 9/RAE, 10/L} {
            \draw (D.pin \i) -- ++(-0.5,0) node[left]{\pin};
        }

        % Right Pins (11-20)
        \foreach \i/\pin in {20/VSS, 19/RD7, 18/RD6, 17/RD5, 16/RD4, 15/RD3, 14/RD2, 13/RD1, 12/RD0, 11/R} {
            \draw (D.pin \i) -- ++(0.5,0) node[right]{\pin};
        }
    \end{circuitikz}
    \caption{Pinout of HUPUF2X8A - DIP-20 Package}
\end{figure}